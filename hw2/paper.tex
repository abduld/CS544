%%%%%%%%%%%%%%%%%%%%%%%%%%%%%%%%%%%%%%%%%
% Arsclassica Article
% LaTeX Template
% Version 1.0 (21/4/14)
%
% This template has been downloaded from:
% http://www.LaTeXTemplates.com
%
% Original author:
% Lorenzo Pantieri (http://www.lorenzopantieri.net) with extensive modifications by:
% Vel (vel@latextemplates.com)
%
% License:
% CC BY-NC-SA 3.0 (http://creativecommons.org/licenses/by-nc-sa/3.0/)
%
%%%%%%%%%%%%%%%%%%%%%%%%%%%%%%%%%%%%%%%%%

%----------------------------------------------------------------------------------------
%	PACKAGES AND OTHER DOCUMENT CONFIGURATIONS
%----------------------------------------------------------------------------------------

\documentclass[
10pt, % Main document font size
letterpaper, % Paper type, use 'letterpaper' for US Letter paper
oneside, % One page layout (no page indentation)
%twoside, % Two page layout (page indentation for binding and different headers)
headinclude,footinclude, % Extra spacing for the header and footer
BCOR5mm, % Binding correction
]{scrartcl}

\input{structure.tex} % Include the structure.tex file which specified the document structure and layout


\usepackage{listings}
\usepackage{xcolor}
\usepackage{color}

\definecolor{mygray}{rgb}{0.4,0.4,0.4}
\definecolor{mygreen}{rgb}{0,0.8,0.6}
\definecolor{myorange}{rgb}{1.0,0.4,0}

\lstset{language=C++,
       basicstyle=\ttfamily\scriptsize,
       keywordstyle=\color{blue}\ttfamily,
       stringstyle=\color{red}\ttfamily,
       commentstyle=\color{green}\ttfamily,
       numbers=left,
       numbersep=5pt,
       numberstyle=\tiny\color{mygray},
       breaklines=true
      }

\newcommand{\fix}[1]{\texttt{\small #1}}

\hyphenation{Fortran hy-phen-ation} % Specify custom hyphenation points in words with dashes where you would like hyphenation to occur, or alternatively, don't put any dashes in a word to stop hyphenation altogether

%----------------------------------------------------------------------------------------
%	TITLE AND AUTHOR(S)
%----------------------------------------------------------------------------------------

\title{\normalfont\spacedallcaps{CS544: MP2}} % The article title

\author{\spacedlowsmallcaps{Abdul Dakkak}} % The article author(s) - author affiliations need to be specified in the AUTHOR AFFILIATIONS block

\date{} % An optional date to appear under the author(s)

%----------------------------------------------------------------------------------------

\begin{document}

%----------------------------------------------------------------------------------------
%	HEADERS
%----------------------------------------------------------------------------------------

\renewcommand{\sectionmark}[1]{\markright{\spacedlowsmallcaps{#1}}} % The header for all pages (oneside) or for even pages (twoside)
%\renewcommand{\subsectionmark}[1]{\markright{\thesubsection~#1}} % Uncomment when using the twoside option - this modifies the header on odd pages
\lehead{\mbox{\llap{\small\thepage\kern1em\color{halfgray} \vline}\color{halfgray}\hspace{0.5em}\rightmark\hfil}} % The header style

\pagestyle{scrheadings} % Enable the headers specified in this block

%----------------------------------------------------------------------------------------
%	TABLE OF CONTENTS & LISTS OF FIGURES AND TABLES
%----------------------------------------------------------------------------------------

\maketitle % Print the title/author/date block


\begin{figure}[h]
\centering
\subfloat[$C \in (0, 0.1)$]{\includegraphics[width=.3\columnwidth]{figs/6.pdf}} \quad%
\subfloat[$C \in (0, 1)$]{\includegraphics[width=.3\columnwidth]{figs/5.pdf}} \quad%
\subfloat[$C \in (0, 10)$]{\includegraphics[width=.3\columnwidth]{figs/7.pdf}} \quad\\
\subfloat[$C \in (0, 100)$]{\includegraphics[width=.3\columnwidth]{figs/4.pdf}} \quad%
\subfloat[$C \in (0, 5000)$]{\includegraphics[width=.3\columnwidth]{figs/1.pdf}} \quad%
\subfloat[$C \in (0, 10000)$]{\includegraphics[width=.3\columnwidth]{figs/3.pdf}} \quad
\caption{This shows the effects of modifying the $C$ value on accuracy for the MONK dataset. As can be seen, the accuracy follows an almost uniform distribution.}
\label{fig:c}
\end{figure}


The short answer is no. 
Taking $169$ training examples from the MONK dataset --- which contains $2$ classes and $6$ features --- and performing a linear SVM while varying the $C$ value we get the accuracies shown in Figure~\ref{fig:c}.
As can be seen, the accuracies are almost uniformly distributed --- achieving between $0.7-0.85$ accuracy for almost all $C$ values.
So, while the entire regularization path can be computed, it is seldom useful in practice.
One can use grid search which is both easier to implement and can be parametrized more carefully -- parameterizing for compute complexity versus accuracy.

The liblinear package, for example, advocate the use of grid search because (1) it is easier to implement and (2) we are more satisfied if we ``think'' we have done an exhaustive search. They take large multiples of $C$, $C \in [10, 10^2, 10^4, \ldots]$, because the algorithm's behaves similar at $C$ vs $C + \epsilon$.

%----------------------------------------------------------------------------------------

\end{document}